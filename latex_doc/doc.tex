% vim: set tw=72 syn=tex ff=unix:

\documentclass[12pt]{article}

\usepackage{graphicx}
\usepackage{paralist}
\usepackage[brazil]{babel}
\usepackage[utf8]{inputenc}
\usepackage{subfigure}
\usepackage{pbox}
\usepackage{tasks}
\usepackage{tcolorbox}
\usepackage{minted}
\usepackage{hyperref}
\usepackage{xcolor}
\usepackage{indentfirst}

\setlength{\parskip}{1em}

\sloppy

%------------------------------------------------------------------------- 

\title{Trabalho Final de ICC (2017/02)}

\author{Pablo Cecilio, Marco Antônio, Lucas Souza}


\date{}
%------------------------------------------------------------------------- 


\begin{document}
\maketitle

\section{Introdução}

O trabalho consiste no processamento e manuseio de uma base de dados usando a programação em bash.

Para esse processamento foi utilizado um script contendo os comandos necessários para executar ações especificas requisitadas pelos itens abaixo descritos.

\section{Pré-processamento}

Para realizar as extrações solicitadas, os dois arquivos ``title.basics.tsv'' e ``title.ratings.tsv'' foram pré-processados utilizando o comando ``join'',  que concatena linhas de dois arquivos através de uma coluna em comum.

\begin{minted}[breaklines]{bash}
join -t $'\t' -o 2.1,2.2,2.3,2.4,2.5,2.6,2.7,2.8,2.9,1.2,1.3 title.ratings.tsv title.basics.tsv > ../$OUTPUTDIR/titles.tsv
\end{minted}

Após a execução do comando, o arquivo ``titles.tsv'' foi gerado na pasta de saída passada por argumento na chamada do script.

\pagebreak

Para finalizar o pre-processamento, o comando ``sed'' foi utilizado, removendo a primeira linha do arquivo ``titles.tsv'' e gerando um novo arquivo,``titles.all.tsv''. 

\begin{minted}[breaklines]{bash}
sed '1d' ../$OUTPUTDIR/titles.tsv > ../$OUTPUTDIR/titles.all.tsv
\end{minted}

De modo a conferir se as saídas foram geradas corretamente, o comando ``wc -l'' foi executado ao final do processo para comparação de seus resultados com o numero de linhas entre os arquivos ``titles.ratings.tsv'' e o arquivos gerados, ``titles.tsv'' e ``titles.all.tsv''.

\begin{minted}[breaklines]{bash}
echo -e "#Linhas nos arquivos originais:"
wc -l ../$IMDBDIR/title.basics.tsv
wc -l ../$IMDBDIR/title.ratings.tsv
echo -e "\n#Linhas nos arquivos gerados:"
wc -l titles.tsv
wc -l titles.all.tsv
\end{minted}

\section{Extrações}

A extração dos itens 1, 2, 3, 4, 5, 6, 7, 8, 9, 10, 11, 12, 16, 17 e 18 foram realizadas. Algumas extrações diferem do gabarito parcial, são elas: itens 10 e 16. Existem duvidas quanto a precisão das saídas dos itens 3, 4 e 5.

\subsection*{Item 1}

O comando ``cut'' foi utilizado para selecionar a segunda coluna em ``title.all.tsv''. Após essa ação, foi realizado um pipe para ``sort'' afim de ordenar essa seleção e em seguida outro pipe para ``uniq'', com isso listando as entradas únicas como saída para o arquivo ``out1''.

\pagebreak

\begin{minted}[breaklines]{bash}
cut -f 2 titles.all.tsv | sort | uniq | tee out1 
\end{minted}

\begin{table}[!h]
    \begin{tabular}{ c c c c }
        \textsc{Entrada} & \textsc{cut -f 2} & \textsc{sort} & \textsc{uniq} \\ 
        \hline
        F A X & A & A & A \\ 
        G B Y & B & A & B \\
        H A Z & A & B & C \\
        K C U & C & C &   \\
    \end{tabular}
\end{table}

\subsection*{Item 2}

Utilizou-se o comando ``awk'' com uma variável como contador, essa por sua vez foi condicionada a aumentar seu valor caso as colunas 3 e 4 fossem iguais. A saída foi impressa em arquivo com o resultado da variável.

\begin{minted}[breaklines]{bash}
awk -F"\t" '{if ($3 == $4) s += 1} END{print s}' titles.all.tsv | tee out2
\end{minted}

\subsection*{Item 3}

Por meio do comando ``awk'', comparou-se a coluna 6 com os valores 1970 a 2000, e para cada resultado positivo uma variável somava a nota dos títulos a um total, enquanto outra variável recebia a quantidade de filmes da condição. Ao final da execução, o total das notas foi dividido pela quantidade de títulos, obtendo-se a media.

\begin{minted}[breaklines]{bash}
awk -F"\t" '{if (1969 < $6  && $6 < 2001){s += $10; t+= 1}} END{printf("%.4f\n", s/t)}' titles.all.tsv | tee out3
\end{minted}

Foi observado uma discrepância entre os valores gerados em sistemas diferentes, que foi atribuída a erros com operações com ponto flutuante devido a diferentes versões ou funções do bash. Como solução, um algorítimo foi acrescentado de forma comentada no script para demonstrar a correção por meio do uso de variáveis e pelo comando ``bc''.

\begin{minted}[breaklines]{bash}
# soma=$(awk -F"\t" '{if ($6>=1970 && $6<=2000){print $10}}' titles.all.tsv | paste -sd+ - | bc) 
# total=$(awk -F"\t" '{if ($6>=1970 && $6<=2000){count++}} END{print count}' titles.all.tsv)
# bc <<< "scale=10;$soma / $total"
\end{minted}

\subsection*{Item 4}

Segue a mesma lógica e princípio do item 3, apresentando a mesma solução e o mesmo erro relatado no item anterior.

\begin{minted}[breaklines]{bash}
awk -F"\t" '{if (1999 < $6  && $6 < 2017){s += $10; t+= 1}} END{printf("%.4f\n", s/t)}' titles.all.tsv | tee out4
\end{minted}

\subsection*{Item 5}

A coluna 9 do arquivo foi separada, e com uma inversão de busca (``grep -v'') eliminou-se as linhas que possuem `,' e `\textbackslash	N'. Após o pipe, a saída foi organizada (``sort'') por ordem alfabética e as entradas repetidas foram removidas com ``uniq'. Por fim, o resultado foi gerado contando-se as linhas dessa saída com o comando ``wc -l'', conforme o proposto.

Diversos erros de  sintaxe foram apresentados no desenvolvimento ao usar o ``grep -v'' no \textbackslash N sem utilizar \textbackslash\textbackslash .

Como detalhe a parte do proposto, foi acrescentado um ``tr -d'' para cortar \textit{whitespaces} gerados pela saída do comando ``wc'' no macOS. Esses espaços em branco eram antes gravados na saída. 

\begin{minted}[breaklines]{bash}
cut -f 9 titles.all.tsv | grep -v "," | grep -v "\\N" | sort | uniq | wc -l | tr -d ' ' | tee out5 
\end{minted}

\subsection*{Item 6}

Utilizou-se ``cut -f 9'' para separar a coluna 9 da base de dados no arquivo, após essa ação, foi feita a busca pela \textit{string}  ``Action''. O resultado foi gerado ao contar as linhas desse saída.

\begin{minted}[breaklines]{bash}
cut -f 9 titles.all.tsv | grep "Action" | wc -l | tr -d ' ' | tee out6
\end{minted}

\subsection*{Item 7}

O retorno da saída foi gerado pelo ``awk'', que tinha como condição encontrar a \textit{string} ``Adventure'' na coluna 9 e o ano maior que 2005, ignorava as entradas com ``\textbackslash N''. Novamente foi usado o ``wc l'' como contador.

\begin{minted}[breaklines]{bash}
awk -F"\t" '$9 ~ /Adventure/ && $6 != "\\N" && $6 >= 2005 {print $6,"\t",$9,"\t",$3}' titles.all.tsv | sort | wc -l | tr -d ' ' | tee out7
\end{minted}

\subsection*{Item 8}

Usa a mesma lógica apresentada no item 7, porém, com condição e operadores lógicos diferentes para a solução.

\begin{minted}[breaklines]{bash}
awk -F"\t" '{if (($9 ~ /Fantasy/ || $9 ~ /Sci-Fi/) && ($6 != "\\N" && $6 >= 2010)) print $6,"\t",$9,"\t",$3}' titles.all.tsv | sort | wc -l | tr -d ' ' | tee out8
\end{minted}

\subsection*{Item 9}

Uma variável foi utilizada para guardar o resultado gerado pela seleção e contagem de itens com o ano igual a 1970 (Contagem feita no mesmo procedimento de itens anteriores). Após o valor ter sido gerado, a razão foi obtida pela divisão desse valor pelo número total de títulos na base já guardados em uma variável anterior no script . Para essa operação ``bc''  foi utilizado a fim de gerar o número de ponto flutuante.

\begin{minted}[breaklines]{bash}
item9=$(awk -F"\t" '{if ($6 == 1970 && $6 != "\\N") print $6}' titles.all.tsv | wc -l | tr -d ' ')
bc <<< "scale=5;$item9 / $titulos" | tee out9
\end{minted}

\subsection*{Item 10}

Foi usada a mesma solução do item 9, porém um laço foi utilizado para gerar por cada ano a razão desse ano pelo total de títulos entre 1971 e 2016.

Uma variável foi gerada para guardar cada ano a ser utilizado no laço, e esse usava o ano dessa variável a fim de selecionar o total de títulos e obter a cada passagem a razão respectiva, já guardada em outra variável (``\$range'').

\begin{minted}[breaklines]{bash}
anos=$(cut -f 6 titles.all.tsv | sort | uniq | sed '$d')
range=$(cut -f 6 titles.all.tsv | awk '$NF >= 1971 && $NF <= 2016' | wc -l | tr -d ' ')
echo Titulos produzidos no itervalo 1971-2016: $range
for i in $anos;
	do
	item10=$(cut -f 6 titles.all.tsv | grep -c "$i")
	media10=$(bc <<< "scale=5;$item10 / $range")
	echo -e $i"\t"$media10 | tee -a out10
	item10=
	done
\end{minted}

Há dúvidas a respeito dos resultados gerados, mesmo esses sendo obtidos pelo comando ``bc''. Os números gerados por meio do laço não coincidem com o gabarito parcial do problema proposto.

\subsection*{Item 11}

Após separar a coluna de ``genres'', a busca por  ``,''  foi invertida gerando o total dado pelas contagem de linhas desse resultado. Como padrão, ``\textbackslash N'' também foi ignorado quando utilizado o mesmo método de busca invertida.

\begin{minted}[breaklines]{bash}
cut -f 9 titles.all.tsv | grep -v "," | grep -v '\\N' | wc -l | tr -d ' ' | tee out11
\end{minted}

\subsection*{Item 12}

Por meio de ``cut'',  separo-se a coluna 9. Após essa ação, o comando ``tr''  foi usado para trocar as ``,'' por  ``\textbackslash n'', separando assim as linhas que possuíam mais de um gênero.  Logo após a ordenação dos itens através do ``sort'', o ``uniq -c'' foi  foi usado para a contagem de itens iguais.

Para a finalizar a extração, uma busca invertida foi usada em ``\textbackslash N'' , com a saída desse comando sendo ordenada com o ``sort'' reverso através do pipe.

\begin{minted}[breaklines]{bash}
cut -f 9 titles.all.tsv | tr -s "," "\n" | sort | uniq -c | grep -v "\N" | sort -g -r | tr -d [:digit:] | tr -d ' ' | head -n 5 | tee out12
\end{minted}

\textit{Nota: Para fins de formatação foi acrescentado o codigo abaixo, que remove os números e espaços da saída.}

\begin{minted}[breaklines]{bash}
tr -d [:digit:] | tr -d ' '
\end{minted}

\pagebreak

\subsection*{Item 16}

Utiliza a mesma lógica apresentada nos itens 9 e 10 para gerar o resultado.

\begin{minted}[breaklines]{bash}
item16a=$(cut -f 8 titles.all.tsv | grep -v "\\N" | awk '$NF >= 80 && $NF <= 120' | wc -l | tr -d ' ')
item16b=$(cut -f 8 titles.all.tsv | grep -c -v "\\N")
bc <<< "scale=5;$item16a / $item16b" | tee out16
\end{minted}

\subsection*{Item 17}

Foi usado o condicionamento no ``awk'' para gerar uma saída a fim de ordena-la com ``sort'' pelo parâmetro da segunda coluna. Por fim, o comando ``head'' foi utilizado para listar os dez primeiros itens da saída 

\begin{minted}[breaklines]{bash}
awk -F"\t" '{if(($9 ~ /Action/) && ($6 >= 2005 && $6 != "\N") && ($2 ~ /movie/) && ($11 > 100)) print $3,"\t",$10}' titles.all.tsv | sort -r -g -t$'\t' -k2 | head -n 10 | tr '\t' ' ' | tee out17
\end{minted}

\textit{Nota: Usou-se ``tr'' para a retirada de espaços para o pipe no arquivo de saída.}

\subsection*{Item 18}

Usa a mesma lógica apresentada no item 17, porém, com condição e operadores lógicos diferentes para a solução.

\begin{minted}[breaklines]{bash}
awk -F"\t" '{ if (($2 ~ /movie/) && ($9 ~ /Comedy/) && ($11 > 100) && ($8 > 200)) print $3,"\t",$10}' titles.all.tsv | sort -r -g -t$'\t' -k2 |  head -n 5 | tr '\t' ' ' | tee out18
\end{minted}

\section{Organização do trabalho}

O processo de colaboração do trabalho foi realizado e documentado utilizando o \textit{GitHub} como plataforma. \url{https://github.com/Durfan/ICC_TP/}

A comunicação imediata, assim como o planejamento, foi feito de forma mais informal por meio do \textit{WhatsApp}. Sendo realizado através desse aplicativo a divisão e designação de tarefas por demanda ou por forma voluntária.

\begin{table}[!h]
    \begin{tabular}{p{5cm}p{7.5cm}}
    
        \textsc{Membro}      & \textsc{Atividades} \\ 
	    \hline
        Pablo Cecilio & pré-processamento; script.sh; itens 1, 7, 8, 9, 10, 16; documentação \\ 
        Marco Antônio & itens 2, 3, 4, 5, 6 e 11 \\
        Lucas Souza & itens 12, 17 e 18 \\
        Arthur Rocha & --- \\

    \end{tabular}
\end{table}

\section{Conclusão}

A falta de uma documentação bash didática e menos técnica dificultou e atrasou o desenvolvimento, sendo que a pesquisa realizada para resolver cada problema enunciado  foi feita na maior parte por soluções diversas encontradas em fóruns e sites como o stackoverflow.com. Somando-se a isso, a maior dificuldade encontrada foi em relação a sintaxe dos comandos, esses que por sua vez variam na forma de suas expressões lógicas e possuem particularidades tais como o uso de aspas simples ou duplas. Ex: O uso de \textit{regexp} para o comando egrep (grep -E), que podia realizar com exatidão metade das buscas enunciadas, porém, sua implementação foi descartada por problemas na sintaxe não correspondendo a solução.

Durante o desenvolvimento também foram encontradas discrepâncias em sistemas diferentes com operações envolvendo ponto flutuante. Sendo que, particularmente, algorítimos envolvendo divisões em sistemas 64bits retornam valores diferentes em relação a outros sistemas, diferindo do problema enunciado. Além disso, o tempo de execução do script varia não apenas devido as especificações da máquina, mas também devido a diferentes versões do bash e ao módulo math incorporado.

\pagebreak

\section{Extra}

\begin{figure}[h]
    \centering
    \includegraphics[width=65mm]{imagens/pablo.jpg}
    \includegraphics[width=65mm]{imagens/marco.jpg}
    \includegraphics[width=65mm]{imagens/lucas.jpg}
    \includegraphics[width=65mm]{imagens/blank.jpg}
\end{figure}


\end{document}
