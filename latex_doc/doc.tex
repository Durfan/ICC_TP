% vim: set tw=72 syn=tex ff=unix:

\documentclass[12pt]{article}

\usepackage{graphicx}
\usepackage{paralist}
\usepackage[brazil]{babel}
\usepackage[utf8]{inputenc}
\usepackage{subfigure}
\usepackage{pbox}
\usepackage{tasks}
\usepackage{tcolorbox}
\usepackage{minted}
\usepackage{hyperref}
\usepackage{xcolor}
\usepackage{indentfirst}

\sloppy

%------------------------------------------------------------------------- 

\title{Trabalho Final de ICC (2017/02)}

\author{Pablo Cecilio, Marco Antônio, Lucas Souza}


\date{}
%------------------------------------------------------------------------- 


\begin{document}
\maketitle

\section{Introdução}

O trabalho consiste no processamento e manuseio entre bases de dados usando a programação em bash. Sendo esse processado através do uso de um script contendo os comandos necessários para executar diversas ações especificas entre as bases de dados originais, gerando a saída requisitada pelos itens abaixo descritos.

\section{Pré-processamento}

Para realizar as extrações solicitadas os dois arquivos ``title.basics.tsv'' e ``title.ratings.tsv'' foram processados utilizando o comando ``join'', que concatena linhas de dois arquivos através de uma coluna em comum.

\begin{minted}[breaklines]{bash}
join -t $'\t' -o 2.1,2.2,2.3,2.4,2.5,2.6,2.7,2.8,2.9,1.2,1.3 title.ratings.tsv title.basics.tsv
\end{minted}

De modo a conferir se a saída foi gerada corretamente, o comando ``wc -l'' foi executado ao final do processo para comparação visual de linhas entre os arquivos ``titles.ratings.tsv'' e o arvivo gerado ``titles.tsv''.

\begin{minted}[breaklines]{bash}
wc -l title.ratings.tsv
wc -l titles.tsv
\end{minted}

Finalizando o pre-processamento, o comando ``sed'' foi utilizado para remover a primeira linha do arquivo ``titles.tsv'', e gerar um novo arquivo ``titles.all.tsv''. Novamente a saída foi verificada com a contagem das respectivas linhas através do comando ``wc -l''.

\begin{minted}[breaklines]{bash}
sed '1d' titles.tsv > titles.all.tsv
wc -l titles.all.tsv
\end{minted}

\section{Extrações}

A extração dos itens 1, 2, 3, 4, 5, 6, 7, 8, 9, 10, 11 e 16 foram realizadas.

\subsection*{Item 1}

\noindent\emph{Liste os tipos de titulo (titleType) únicos existentes e imprima em ordem lexicográfica.}
\vspace{1em}

Para isso foi utilizado o comando ``cut'' para selecionar a coluna 2 em ``title.all.tsv'', após esse comando foi realizado um pipe para um ``sort'' afim de ordenar a seleção e em seguida outro pipe para ``uniq'', para que com isso se pudesse listar as entradas únicas da seleção como saída para o arquivo.

\begin{minted}[breaklines]{bash}
cut -f 2 titles.all.tsv | sort | uniq > out1.txt
\end{minted}

\begin{table}[!h]
    \begin{tabular}{ c c c c }
        \textsc{Entrada} & \textsc{cut -f 2} & \textsc{sort} & \textsc{uniq} \\ 
        \hline
        F A X & A & A & A \\ 
        G B Y & B & A & B \\
        H A Z & A & B & C \\
        K C U & C & C &   \\
    \end{tabular}
\end{table}

\subsection*{Item 2}

\noindent\emph{Quantos títulos tem o primaryTitle e o originalTitle iguais?}
\vspace{1em}

Foi utilizado o comando ``awk'' com uma variável como contador, essa por sua vez foi condicionada a aumentar seu valor caso as colunas 3 e 4 fossem iguais. A saída foi impressa em arquivo com o resultado da variável.

\begin{minted}[breaklines]{bash}
awk -F"\t" '{if ($3 == $4) s += 1}END{print s}' titles.all.tsv > out2.txt
\end{minted}

\clearpage

\subsection*{Item 3}

\noindent\emph{Qual a media das avaliações feitas entre os anos 1970 e 2000?}
\vspace{1em}

Atraves do comando awk compara se a coluna 6 está entre os valores (1970 a 2000), recebe em s a nota do filme, e em t a quantidade de filmes, imprime em out3 a media das notas.

Erro nos itens 3 e 4 por ter que colocar duas condições para uma mesma variável separadas.

\begin{minted}[breaklines]{bash}
awk -F"\t" '{if (1969 < $6  && $6 < 2001){s += $10; t+= 1}}END{print s/t}' titles.all.tsv > out3.txt
\end{minted}

\subsection*{Item 4}

\noindent\emph{Qual a media das avaliações feitas entre os anos 2000 e 2016?}
\vspace{1em}

Compara se a coluna 6 está entre os valores (2000 a 2016), recebe em s a nota do filme, e em t a quantidade de filmes, imprime em out4.txt a media das notas.

Erro nos itens 3 e 4 por ter que colocar duas condições para uma mesma variável separadas.

\begin{minted}[breaklines]{bash}
awk -F"\t" '{if (1999 < $6  && $6 < 2017){s += $10; t+= 1}}END{print s/t}' titles.all.tsv > out4.txt
\end{minted}

\subsection*{Item 5}

\noindent\emph{Quantos gêneros (genres) únicos existem, sem considerar o marcador de nenhum gênero (``\textbackslash N'')?}
\vspace{1em}

Separa a coluna 9 do arquivo, elimina (com grep -v) as linhas que possuem `,' e `\textbackslash	N', organiza (com sort) por ordem alfabética, elimina (com uniq) as entradas repetidas, conta as linhas com wc -l e imprime o numero de linhas no out5.txt.

Erro no item 5 sintaxe para usar o grep -v no \textbackslash N ter que utilizar duas \textbackslash .

\begin{minted}[breaklines]{bash}
cut -f9 titles.all.tsv | grep -v "," | grep -v '\\N' | sort | uniq | wc -l > out5.txt
\end{minted}

\subsection*{Item 6}

\noindent\emph{Quantos títulos foram classificados como ação (Action)?}
\vspace{1em}

Separa a coluna 9 do arquivo, separa todas linhas que possuem a palavra Action, conta as linhas e printa o numero em out6.txt

\subsection*{Item 7}

\noindent\emph{Quantos títulos de aventura (Adventure) foram produzidos desde 2005 (startYear)?}
\vspace{1em}

Separa a coluna 9 do arquivo, separa todas linhas que possuem a palavra Action, conta as linhas e printa o numero em out6.txt

\subsection*{Item 8}

\noindent\emph{Quantos títulos de fantasia (Fantasy) e ficção (Sci-Fi) foram produzidos desde 2010?}
\vspace{1em}

Separa a coluna 9 do arquivo, separa todas linhas que possuem a palavra Action, conta as linhas e printa o numero em out6.txt

\subsection*{Item 9}

\noindent\emph{Obtenha a razão de títulos produzidos (startYear) em 1970 em relação ao total de títulos na base.}
\vspace{1em}

Separa a coluna 9 do arquivo, separa todas linhas que possuem a palavra Action, conta as linhas e printa o numero em out6.txt

\begin{minted}[breaklines]{bash}
cut -f9 titles.all.tsv | grep "Action"| wc -l > out6.txt
\end{minted}

\subsection*{Item 10}

\noindent\emph{Obtenha a razão de títulos produzidos em cada ano (startYear) em relação ao total de títulos na base no intervalo de 1971 a 2016}
\vspace{1em}

Separa a coluna 9 do arquivo, separa todas linhas que possuem a palavra Action, conta as linhas e printa o numero em out6.txt

\subsection*{Item 11}

\noindent\emph{Quantos filmes tiveram apenas um único gênero?}
\vspace{1em}

Separa a coluna 9, elimina as linhas que possuem `,' e `\textbackslash N' sobrando somente os generos unicos e conta as linhas com wc -l

\begin{minted}[breaklines]{bash}
cut -f9 titles.all.tsv | grep -v "," | grep -v '\\N' | wc -l
\end{minted}

\subsection*{Item 16}

\noindent\emph{Imprima a razão de títulos, em relação ao total, que tenha duração (runtimeMinutes) entre 80 e 120 minutos (80 e 120 também são considerados).}
\vspace{1em}

Separa a coluna 9 do arquivo, separa todas linhas que possuem a palavra Action, conta as linhas e printa o numero em out6.txt

\section{Organização do trabalho}

O processo de colaboração do trabalho foi realizado e documentado utilizando o GitHub como plataforma. \url{https://github.com/Durfan/ICC_TP/}

A comunicação imediata assim como o planejamento foi feito de forma mais informal através do WhatsApp. Sendo realizado através desse aplicativo a divisão e designação de tarefas por demanda ou por forma voluntaria.

\begin{table}[!h]
    \begin{tabular}{p{5cm}p{7.5cm}}
    
        \textsc{Membro}      & \textsc{Atividades} \\ 
	    \hline
        Pablo Cecilio & pré-processamento; script.sh; itens 1, 7, 8, 9, 10, 16; documentação \\ 
        Marco Antônio & itens 2, 3, 4, 5, 6 e 11 \\
        Lucas Souza & --- \\
        Arthur Rocha & --- \\

    \end{tabular}
\end{table}

\section{Conclusão}

A falta de uma documentação bash didática e menos técnica dificultou e atrasou o desenvolvimento(...)

\pagebreak
\section{Extra}

\begin{figure}[h]
    \centering
    \includegraphics[scale=0.3]{imagens/extra.jpg}
\end{figure}


\end{document}