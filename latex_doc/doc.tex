% vim: set tw=72 syn=tex ff=unix:

\documentclass[12pt]{article}

\usepackage{graphicx}
\usepackage{paralist}
\usepackage[brazil]{babel}
\usepackage[utf8]{inputenc}
\usepackage{subfigure}
\usepackage{pbox}
\usepackage{tasks}
\usepackage{tcolorbox}
\usepackage{minted}
\usepackage{xcolor}

\sloppy

%------------------------------------------------------------------------- 

\title{Trabalho Final de ICC (2017/02)}

\author{Pablo Cecilio, Marco Antônio, Lucas Souza, Arthur Rocha}


\date{}
%------------------------------------------------------------------------- 


\begin{document} 
\maketitle

\section{Introdução}

\paragraph{}
O trabalho consiste no processamento e manuseio entre bases de dados usando a programação em bash. Sendo esse processado através do uso de um script contendo os comandos necessários para executar diversas ações especificas entre as bases de dados originais, gerando a saída requisitada pelos itens abaixo descritos.

\section{Pré-processamento}

\paragraph{}
Afim de concatenar os dois arquivos foi utilizado o comando ``join'', uma linha de comando que concatena linhas de dois arquivos através de uma coluna em comum.
\begin{minted}[breaklines]{bash}
join -t $'\t' -o 2.1,2.2,2.3,2.4,2.5,2.6,2.7,2.8,2.9,1.2,1.3 title.ratings.tsv title.basics.tsv
\end{minted}

\paragraph{}
De modo a conferir se essa saída foi gerada corretamente, foi utilizado um ``wc -l'' comparando ``titles.ratings.tsv'' e o arvivo gerado ``titles.tsv''.
\begin{minted}[breaklines]{bash}
wc -l title.ratings.tsv
wc -l titles.tsv
\end{minted}

\paragraph{}
Para finalizar o pre-processamento, o comando ``sed'' foi utilizado afim de remover a primeira linha do arquivo ``titles.tsv'' para gerar um novo arquivo, ``titles.all.tsv''. Novamente a saída foi verificada com a contagem das respectivas linhas através do comando ``wc -l''.
\begin{minted}[breaklines]{bash}
sed '1d' titles.tsv > titles.all.tsv
wc -l titles.all.tsv
\end{minted}

\section{Extrações}

\paragraph{}
A extração dos itens 1, 2, 3, 4, 5, 6 e 11 foram realizdas.

\subsection*{Item 1}

\paragraph{}
Utilizou-se ``cut'' para selecionar a coluna 2 em ``title.all.tsv'', após esse comando foi realizado um pipe para um ``sort'' afim de ordenar a seleção, em seguida outro pipe para ``uniq'' e com isso listar as entradas únicas dessa seleção como saída para o arquivo.

\begin{table}[!htb]
    \begin{tabular}{ c c c c }
        \textsc{Entrada} & \textsc{cut -f 2} & \textsc{sort} & \textsc{uniq} \\ 
        \hline
        F A X & A & A & A \\ 
        G B Y & B & A & B \\
        H A Z & A & B & C \\
        K C U & C & C &   \\
    \end{tabular}
\end{table}

\section{Organização do trabalho}

\emph{Explique nessa seção como o trabalho foi distribuído para o grupo.
Por exemplo, toda a equipe fez todos os itens junto ou as tarefas foram 
distribuídos para os membros da equipe? Crie uma tabela onde a linha é o
nome do membro da equipe e a coluna são as atividades desempenhadas.
Essa seção será utilizada para a avaliação e arguição. Portanto, sejam
honestos.}

\begin{table}[!htb]
    \begin{tabular}{p{5cm}p{7.5cm}}
    
        \textsc{Membro}      & \textsc{Atividades} \\ 
	    \hline
        Pablo Cecilio & pre-processamento; script.sh; item 1; documentação \\ 
        Marco Antônio & itens 2,3,4,5,6,11\\
        Lucas Souza & --- \\
        Arthur Rocha & --- \\

    \end{tabular}
\end{table}

\section{Conclusão}

\emph{Discuta aqui os principais desafios enfrentados no trabalho e como
cada desafio foi superado. Discuta também o aprendizado 
da equipe durante o trabalho.}

\pagebreak
\section{Extra}

\begin{figure}[h]
    \centering
    \includegraphics[scale=0.3]{imagens/extra.jpg}
\end{figure}


\end{document}
