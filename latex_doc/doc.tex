% vim: set tw=72 syn=tex ff=unix:

\documentclass[12pt]{article}

\usepackage{graphicx}
\usepackage{paralist}
\usepackage[brazil]{babel}
\usepackage[utf8]{inputenc}
\usepackage{subfigure}
\usepackage{pbox}
\usepackage{tasks}
\usepackage{tcolorbox}

\usepackage{minted}
\usepackage{xcolor}

\sloppy

%------------------------------------------------------------------------- 

\title{Trabalho Final de ICC (2017/02)}

\author{Pablo Cecilio, Marco Antônio, Lucas Souza, Arthur Rocha}


\date{}
%------------------------------------------------------------------------- 


\begin{document} 
\maketitle

\section{Introdução}

\paragraph{}
O trabalho consiste no processamento e manuseio entre bases de dados usando a programação em bash. Sendo esse processado através do uso de um script contendo os comandos necessários para executar diversas ações especificas entre as bases de dados originais, gerando a saída requisitada pelos itens abaixo descritos.

\section{Pré-processamento}

\paragraph{}
Afim de concatenar os dois arquivos foi utilizado o comando ``join'', uma linha de comando que concatena linhas de dois arquivos através de uma coluna em comum.
\begin{minted}[breaklines]{bash}
join -t $'\t' -o 2.1,2.2,2.3,2.4,2.5,2.6,2.7,2.8,2.9,1.2,1.3 title.ratings.tsv title.basics.tsv
\end{minted}

\paragraph{}
De modo a conferir se essa saída foi gerada corretamente, foi utilizado um ``wc -l'' comparando ``titles.ratings.tsv'' e o arvivo gerado ``titles.tsv''.
\begin{minted}[breaklines]{bash}
wc -l title.ratings.tsv
wc -l titles.tsv
\end{minted}

\paragraph{}
Para finalizar o pre-processamento, o comando ``sed'' foi utilizado afim de remover a primeira linha do arquivo ``titles.tsv'' para gerar um novo arquivo, ``titles.all.tsv''. Novamente a saída foi verificada com a contagem das respectivas linhas através do comando ``wc -l''.
\begin{minted}[breaklines]{bash}
sed '1d' titles.tsv > titles.all.tsv
wc -l titles.all.tsv
\end{minted}

\section{Extrações}

\emph{Descreva aqui quais itens foram feitos, quais não foram feitos e
quais sua equipe não tem certeza da resposta. Explique os motivos.}

\subsection*{Item 1}

\emph{Discuta aqui cada comando utilizado para resolver o problema. O
exemplo hipotético da seção de ``Entrada e saída'' da especificação
poderia ser explicado da seguinte forma:}\\
\emph{Para contar o número de itens do arquivo ``title.basics.tsv'' nós
utilizamos o comando cat para imprimir o arquivo na saída padrão e
utilizamos o comando wc (parâmetro -l) em cascata (pipe) para contar quantas 
linhas foram impressas, resultando no total de linhas do arquivo
``title.basics.tsv''.}


\subsection*{Item 2}
\ldots

\subsection*{Item 18}

\section{Organização do trabalho}

\emph{Explique nessa seção como o trabalho foi distribuído para o grupo.
Por exemplo, toda a equipe fez todos os itens junto ou as tarefas foram 
distribuídos para os membros da equipe? Crie uma tabela onde a linha é o
nome do membro da equipe e a coluna são as atividades desempenhadas.
Essa seção será utilizada para a avaliação e arguição. Portanto, sejam
honestos.}

\begin{table}[!htb]
    \begin{tabular}{p{5cm}p{7.5cm}}
    
        \textsc{Membro}      & \textsc{Atividades} \\ 
	    \hline
        Pablo Cecilio & pre-processamento; script.sh; item 1; documentação \\ 
        Marco Antônio & ---  \\
        Lucas Souza & ---  \\
        Arthur Rocha & ---  \\

    \end{tabular}
\end{table}

\section{Conclusão}

\emph{Discuta aqui os principais desafios enfrentados no trabalho e como
cada desafio foi superado. Discuta também o aprendizado 
da equipe durante o trabalho.}

\pagebreak
\section{Extra}

\begin{figure}[h]
    \centering
    \includegraphics[scale=0.3]{imagens/extra.jpg}
\end{figure}


\end{document}